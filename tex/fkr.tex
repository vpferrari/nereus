\documentclass[a4paper, twocolumn, 10pt]{book}
\usepackage[brazilian]{babel}
\usepackage[utf8]{inputenc}
\usepackage[margin=15mm]{geometry}

\begin{document}

\tableofcontents

\chapter{Regras}

\section{O que é isso?}

FFKKRR é um Jogo de Representação (RPG) de Fantasia Medieval. Nele, os participantes representam aventureiros que se aventuram em um mundo medieval fantástico, explorando lugares inóspitos, enfrentando ameaças, desvendando mistérios e descobrindo relíquias ancestrais. Suas regras são simples e tudo o que você precisa para jogar, além deste livro, é de dados, lápis e papel.

\section{Árbitro e Jogadores}

Em FFKKRR, um dos participantes será o Árbitro e os demais os Jogadores.

Os Jogadores irão representar seus respectivos Personagens Jogadores (PJs), tomando decisões por eles, declarando o que fazem ou dizem.

O Árbitro é responsável em descrever o ambiente em que os PJs se encontram, apresentar desafios, obstáculos e perigos existentes em uma aventura. O Árbitro também julga o sucesso ou falha das ações que os Jogadores escolheram para os seus personagens. Para isso, na maior parte do tempo, o Árbitro usa o bom senso, mas algumas situações específicas podem exigir jogada de dados e o uso das regras do jogo. Por fim, o Árbitro também representa os Personagens Não Jogadores (PNJs) que são aliados, adversários e monstros que interagem com os PJs dentro da aventura.   
O jogo sempre gira em torno da aventura e dos objetivos dos PJs. O jogo se inicia com a descrição do Árbitro sobre alguma localidade ou evento (normalmente problemático para os PJs). Em um dado momento, a descrição exige que os Jogadores representem seus personagens e tomem decisões pelos seus aventureiros. Neste instante, o Árbitro então pergunta "O que vocês fazem?".

Diante dos problemas, os Jogadores devem tomar decisões para resolver o problema como se fossem os próprios aventureiros. As ações são declaradas e o Árbitro determina sucesso, falha e suas consequências, sendo que algumas situações irão exigir que os participantes joguem dados para observar o desfecho das coisas.

Diante das escolhas e dos resultados, o Árbitro descreve uma nova situação decorrente dos eventos que acabaram de acontecer e o ciclo se repete até que o objetivo da aventura seja alcançado ou até que a sessão se encerre para continuar na sessão seguinte.

\section{Mecânica Central}

Sempre que um PJ tentar algo desafiador, complexo ou perigoso, o Jogador deverá jogar dois dados e somar com uma Habilidade pertinente à tarefa. Se o resultado for igual ou maior que um Valor de Dificuldade estipulado pelo Árbitro, a ação foi bem sucedida. Caso o resultado seja menor que o Valor de Dificuldade, a ação falhou.

O Valor de Dificuldade é sempre estabelecido pelo Árbitro, normalmente variando entre 5 (para ações fáceis e simples) e 13 (para tarefas desafiadoras e heroicas). Você saberá mais sobre que habilidades um personagem pode ter e como determinar os Valores de Dificuldade no capítulo \ref{chap:regras}.

\section{Exemplo de Jogo}

Texto

\chapter{Personagens Aventureiros}

\section{Criando um Aventureiro}

Texto

\section{Resistência e Sorte}

Texto

\section{Habilidades}

Texto

\section{Equipamentos}

Texto

\section{Exemplos de Personagens}

Texto

\chapter{Regras} \lable{chap:regras}

\section{Turnos e Ações}

Texto

\section{Movimentação e Distância}

Texto

\section{Testando suas Habilidades}

Texto

\section{Testando sua Sorte}

\section{Combate}

Texto

\section{Ferimentos e Cura}

Texto

\section{Consequências Extremas}

Texto

\section{Experiência}

Texto: [XPs, Tabela de Efeitos: [Aumento de Resistência, Aumento de Habilidade, Nova Habilidade, Aumento de Sorte]]

\chapter{Magia}

Texto

\section{Conjurando Feitiços}

Texto

\section{Lista de Feitiços Arcanos}

Texto

\section{Lista de Feitiços Sacerdotais}

Texto

\chapter{O Árbitro}

Texto: [Aventura, Campanha, Tarefas do Árbitro, Jogadas de Probabilidades, Lidando com Situações Comuns, PNJs, Improvisando, Atmosfera de Jogo]

\chapter{Monstros}

Texto

\chapter{Tesouros}

Texto

\appendix

\chapter{Masmorras}

Texto

\chapter{Regiões}

Text

\chapter{Povoados}

Text

\chapter{Referências para a Fantasia Medieval}

Text

\end{document}
