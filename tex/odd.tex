\documentclass[a5paper, 10pt]{book}
\usepackage[brazilian]{babel}
\usepackage[utf8]{inputenc}
\usepackage[margin=15mm]{geometry}

\begin{document}

\tableofcontents

\chapter{Introdução}

\section{O que é isso?}

FFKKRR é um Jogo de Representação (RPG) de Fantasia Medieval. Nele, os participantes representam aventureiros que se aventuram em um mundo medieval fantástico, explorando lugares inóspitos, enfrentando ameaças, desvendando mistérios e descobrindo relíquias ancestrais. Suas regras são simples e tudo o que você precisa para jogar, além deste livro, é de dados, lápis e papel.

\section{Árbitro e Jogadores}

Texto

\section{Dados}

Texto

\section{Exemplo de Jogo}

Texto

\chapter{Personagens Aventureiros}

\section{Criando um Aventureiro}

A primeira coisa que os jogadores devem fazer antes do jogo iniciar é criar seus respectivos Personagens Aventureiros. "Criar um personagem" significa determinar suas características de jogo e anotá-las em um papel ou em uma cópia da ficha de personagem (cujo modelo pode ser encontrado no final deste livro).

Quando criar um personagem:

\begin{enumerate}
	\item Determine suas seis Habilidades: Força, Destreza, Constituição, Inteligência, Sabedoria e Carisma. Anote seus valores e modificadores.
	\item Escolha uma Classe de Personagem que melhor combine com o papel que você quer representar. Anote as suas Aptidões e Habilidades Especiais de Classe.
	\item Compre o seu Equipamento inicial.
	\item Anote seu modificador de Aptidão, Classe de Armadura e Pontos de Vida (Máximo e Atual).
\end{enumerate}

\section{Habilidades}

Texto

\section{Classes}

Texto

\section{Equipamento}

Texto

\section{Aptidão}

Texto

\section{Línguas}

Texto


\end{document}

