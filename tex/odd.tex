\documentclass[a5paper, 10pt]{book}
\usepackage[brazilian]{babel}
\usepackage[utf8]{inputenc}
\usepackage[margin=15mm]{geometry}

\begin{document}

\tableofcontents

\chapter{Introdução}

\section{O que é isso?}

FFKKRR é um Jogo de Representação (RPG) de Fantasia Medieval. Nele, os participantes representam aventureiros que se aventuram em um mundo medieval fantástico, explorando lugares inóspitos, enfrentando ameaças, desvendando mistérios e descobrindo relíquias ancestrais. Suas regras são simples e tudo o que você precisa para jogar, além deste livro, é de dados, lápis e papel.

\section{Árbitro e Jogadores}

Texto

\section{Dados}

Texto

\section{Exemplo de Jogo}

Texto

\chapter{Personagens Aventureiros}

\section{Criando um Aventureiro}

A primeira coisa que os jogadores devem fazer antes do jogo iniciar é criar seus respectivos Personagens Aventureiros. "Criar um personagem" significa determinar suas características de jogo e anotá-las em um papel ou em uma cópia da ficha de personagem (cujo modelo pode ser encontrado no final deste livro).

Quando criar um personagem:

\begin{enumerate}
	\item Determine suas seis Habilidades: Força, Destreza, Constituição, Inteligência, Sabedoria e Carisma. Anote seus valores e modificadores.
	\item Escolha uma Classe de Personagem que melhor combine com o papel que você quer representar. Anote as suas Aptidões e Habilidades Especiais de Classe.
	\item Compre o seu Equipamento inicial.
	\item Anote seu modificador de Aptidão, Classe de Armadura e Pontos de Vida (Máximo e Atual).
\end{enumerate}

\section{Habilidades}

Existem seis habilidades básicas de todo personagem aventureiro:

\begin{description}
	\item [Força (For)] representa sua força física bruta.
	\item [Destreza (Des)] representa a sua agilidade, reflexos, coordenação e equilíbrio.
	\item [Constituição (Con)] representa a sua saúde e resistência física.
	\item [Inteligência (Int)] representa o seu intelecto e raciocínio.
	\item [Sabedoria (Sab)] representa o seu senso comum, intuição e percepção.
	\item [Carisma (Car)] representa a sua personalidade, aparência e presença.
\end{description}

Cada habilidade possui um Valor que varia de 2 a 12. No momento da criação do seu personagem, jogue 2d6 e anote o resultado. Repita o processo por mais cinco vezes, a fim de obter seis valores. Assinale cada um dos valores para uma das seus habilidades, de acordo com a sua preferência. Tenha em mente a escolha de qual Classe o seu personagem pertence para tomar decisões apropriadas.

Além do Valor de Habilidade, você também precisa anotar o Modificador para cada uma delas - de acordo com a tabela abaixo:

\begin{tabular}{||c c||} 
 \hline
 Valor de Habilidade & Modificador \\ [0.5ex] 
 \hline
	2 & -3 \\
 \hline
	3 & -2 \\ 
 \hline
	4 & -1 \\
 \hline
	5-7 & +0 \\
 \hline
	8-9 & +1 \\
 \hline
	10-11 & +2 \\ 
 \hline
	12+ & +3 \\
 \hline 
\end{tabular}

\section{Classes}

Texto

\section{Equipamento}

Texto

\section{Aptidão}

Texto

\section{Línguas}

Texto


\end{document}

